%%%%%%%%%%%%%%%%%%%%%%%%%%%%%%%%%%%%%%%%%
% Beamer Presentation
% LaTeX Template
% Version 1.0 (10/11/12)
%
% This template has been downloaded from:
% http://www.LaTeXTemplates.com
%
% License:
% CC BY-NC-SA 3.0 (http://creativecommons.org/licenses/by-nc-sa/3.0/)
%
%%%%%%%%%%%%%%%%%%%%%%%%%%%%%%%%%%%%%%%%%

%----------------------------------------------------------------------------------------
%	PACKAGES AND THEMES
%----------------------------------------------------------------------------------------

\documentclass[svgnames]{beamer}

\mode<presentation> {

% The Beamer class comes with a number of default slide themes
% which change the colors and layouts of slides. Below this is a list
% of all the themes, uncomment each in turn to see what they look like.

%\usetheme{default}
%\usetheme{AnnArbor}
%\usetheme{Antibes}
%\usetheme{Bergen}
%\usetheme{Berkeley}
%\usetheme{Berlin}
%\usetheme{Boadilla}
%\usetheme{CambridgeUS}
%\usetheme{Copenhagen}
%\usetheme{Darmstadt}
%\usetheme{Dresden}
%\usetheme{Frankfurt}
%\usetheme{Goettingen}
%\usetheme{Hannover}
%\usetheme{Ilmenau}
%\usetheme{JuanLesPins}
%\usetheme{Luebeck}
\usetheme{Madrid}
%\usetheme{Malmoe}
%\usetheme{Marburg}
%\usetheme{Montpellier}
%\usetheme{PaloAlto}
%\usetheme{Pittsburgh}
%\usetheme{Rochester}
%\usetheme{Singapore}
%\usetheme{Szeged}
%\usetheme{Warsaw}

% As well as themes, the Beamer class has a number of color themes
% for any slide theme. Uncomment each of these in turn to see how it
% changes the colors of your current slide theme.

%\usecolortheme{albatross}
%\usecolortheme{beaver}
%\usecolortheme{beetle}
%\usecolortheme{crane}
%\usecolortheme{dolphin}
%\usecolortheme{dove}
%\usecolortheme{fly}
%\usecolortheme{lily}
%\usecolortheme{orchid}
%\usecolortheme{rose}
%\usecolortheme{seagull}
%\usecolortheme{seahorse}
%\usecolortheme{whale}
%\usecolortheme{wolverine}

%\setbeamertemplate{footline} % To remove the footer line in all slides uncomment this line
%\setbeamertemplate{footline}[page number] % To replace the footer line in all slides with a simple slide count uncomment this line

%\setbeamertemplate{navigation symbols}{} % To remove the navigation symbols from the bottom of all slides uncomment this line
}

\usepackage{graphicx} % Allows including images
\usepackage{booktabs} % Allows the use of \toprule, \midrule and \bottomrule in tables

\usepackage{amsmath,amsfonts,amssymb,amsthm}
\usepackage{enumerate}
\usepackage{listings}
\usepackage{xcolor}

\newcommand{\code}[1]{{\normalfont\ttfamily #1}}
\newcommand{\mcode}[1]{\text{\code{#1}}}
\newcommand{\red}[1]{{\color{red} #1}}
\newcommand{\blue}[1]{{\color{blue} #1}}

\lstnewenvironment{cxxlisting}{
	\lstset{
		language=c++,
		tabsize=4,
		basicstyle=\ttfamily,
		breaklines=false,
	}
}{}

\lstloadlanguages{Lisp}
\lstnewenvironment{racket}{
	\lstset{
		language={[Auto]Lisp},
		tabsize=4,
		basicstyle=\scriptsize\ttfamily,
		breaklines=true,
		stringstyle=\color{Maroon},
		commentstyle=\color{ForestGreen},
		rulecolor=\color{gray!10},
		keywordstyle=\color{blue},
		morekeywords={define, define-syntax, syntax-rules, begin, let, lang, begin-for-syntax, make-readtable, current-readtable, parameterize, define-values, letrec, call, cc}
	}
}{}

%----------------------------------------------------------------------------------------
%	TITLE PAGE
%----------------------------------------------------------------------------------------

\title[Distributed Systems Recitation]{Distributed Systems Recitation} % The short title appears at the bottom of every slide, the full title is only on the title page

\author{Lamont Nelson} % Your name
\institute[NYU] % Your institution as it will appear on the bottom of every slide, may be shorthand to save space
{
New York University \\ % Your institution for the title page
\medskip
\textit{lamont.nelson@nyu.edu} % Your email address
}
\date{\today} % Date, can be changed to a custom date

\begin{document}


\begin{frame}
\titlepage % Print the title page as the first slide
\end{frame}

\begin{frame}
	\frametitle{Overview} % Table of contents slide, comment this block out to remove it
	\tableofcontents % Throughout your presentation, if you choose to use \section{} and \subsection{} commands, these will automatically be printed on this slide as an overview of your presentation
\end{frame}

\section{Excercises}
\section{Channels in Go}
\section{Review Lab 1}
\section{Lab Questions}

\begin{frame}{Concurrency in Go}
\begin{itemize}
	\item Channels are a FIFO queue abstraction that allow inter-thread communication
	\item Two types: buffered and unbuffered
	\item Buffered channels block the sender if the buffer is full and blocks the receiver if the buffer is empty.
	\item Unbuffered channels block the receiver/sender if there is no data to be sent/received.
	\item Multiple channels can be polled for data using the switch construct.
	\item Channels can be used to replace tradtional concurreny control abstractions such as Mutexes and Semaphores.
	\item Based on the work by C. A. R. Hoare. Communicating Sequential Processes. 1978.
\end{itemize}
\end{frame}

\begin{frame}{Concurrency in Go (cont)}
\begin{itemize}
	\item The language also offers other concurrency control mechanisms in the "sync" and "sync/atomic" packages.
	\item Examples: Mutex, Barriers (WaitGroup), Condition Variables, Atomic Load/Stores
\end{itemize}
\end{frame}

\begin{frame}{Producer/Consumer Problem}
\begin{itemize}
	\item Also called the bounded buffer problem
	\item Setup:
		\begin{itemize}
			\item Fixed size buffer
			\item $N$ Producer threads - put items onto the buffer
			\item $M$ consumer threads - retrieve items from the buffer
		\end{itemize}
	\item Problem: How can we allow the producers to concurrently distribute items to the consumers without overflowing the buffer?
	\item Multiple ways to accomplish this in Go
\end{itemize}
\end{frame}

\begin{frame}{Lab 1 Questions}
\begin{itemize}
	\item How can we determine the current list of workers?
	\item What are the paramaters required to make an RPC call?
	\item What should happen if a worker fails? How do we know a worker has failed?
	\item Other Questions?
\end{itemize}
\end{frame}


\begin{frame}
\Huge{\centerline{The End}}
\end{frame}

\end{document} 